\newcommand{\grafik}[4][0.9]{       % Bild einfügen, [Skalierung], {Dateiname ohne Endung}, {Beschriftung}, {label zum Referenzieren}
    \begin{figure}[ht]          % before: htbp 
        \begin{center}
            \includegraphics[width=#1\columnwidth]{Images/#2}
            \caption{\label{#4} #3}
            
        \end{center}
    \end{figure}
}

\newcommand{\xfig}[4][0.9] {      %xfig figure einfügen, [Skalierung], {Dateiname ohne Endung}, {Beschriftung}, {label}
    \begin{figure}[ht]
        \begin{center}
            \graphicspath{{./}{Images/}}
            \scalebox{#1}{
                \input{Images/#2}
            }
            \caption{\label{#4} #3}
        \end{center}
    \end{figure}
}


\newcommand{\reFLect}{\textit{re\kern-0.07em F\kern-0.07emL\kern-0.29em\raisebox{0.56ex}{ect}}}

\newcommand{\boxit}[1]{\mbox{{\it #1}}}

\newcommand{\hs}[1]{\mbox{\lstinline[basicstyle=\color{textgray}]!#1!}}

\newcommand{\funct}[1]{``#1''}

\newcommand{\begriff}[1]{{\it ``#1''}}
\newcommand{\source}[1]{\mbox{\tt{#1}}}
\newcommand{\hsSource}[1]{\mbox{\tt{#1}}}
\newcommand{\hsSourceLine}[1]{$$\mbox{\tt{#1}}$$}
\newcommand{\zahl}[1]{\mbox{$#1$}}





\newtheorem{theo}{Theorem}
\newtheorem{lem}{Lemma}[section]
\newtheorem{rem}{Remark}
