\section{Einleitung}

Mit dem Aufkommen von Hardwarebeschreibungssprachen (HDL\footnote{Hardware Description Language})wie VHDL\footnote{Very High Speed
  Integrated Circuit Hardware Description Language}, Verilog oder SystemC hat sich der Entwicklungsprozess für Hardware stark dem
für Software angeglichen. Es überrascht nicht, dass es daher die unterschiedlichsten Ansätze gibt, um Konzepte aus der
Softwareentwicklung auch auf den Hardwareentwicklungsprozess zu übertragen. Dabei beeinflusst vor allem die imperative
Programmierung den Hardwareentwicklungsprozess; die genannten HDL's sind imperative Sprachen oder aus solchen entstanden.

Das Forschungsinteresse dieser Dissertation liegt zum einen innerhalb der funktionalen Programmierung und zum anderen im Erstellen
und Entwerfen von Hardwarebeschreibenden domänenspezifischen Programmiersprachen. Haskell wurde als funktionale Wirtssprache aus
verschiedenen Gründen ausgewählt. \emph{1:} Um Haskell herum hat sich die größte Community gebildet, womit keine andere
funktionale Sprache mithalten kann; \emph{2:} mit dem Hindley-Milner Typsystem besitzt Haskell ein sehr gut ausgearbeitetes
Typsystem; und \emph{3:} Haskell ist keine Sprache mit funktionalen Eigenschaften, sondern sie ist eine ``rein'' funktionale
Programmiersprache. 

\section{Bisherige Arbeiten}
Hardwareentwicklung mit funktionalen Sprachen ist keine neue Erfindung und so gibt bis heute viele verschiedene Ansätze. Die
frühen Anfänge gehen bis in die 80'er Jahre zurück. Damals stellte eine Gruppe von Wissenschaftlern um Mary Sheeran eine
funktionalen HDL $\mu$FP\cite{sheeran:muFP} vor. Im selben Jahrzehnt präsentierte John O'Donnell seinen HDL Ansatz,
HDRE\cite{hydra:old} \cite{donnell}. Diese Pioniersarbeiten, Hardwarebeschreibungssprachen innerhalb des funktionalen Paradigmas,
werden heute nicht mehr aktiv entwickelt, sie haben aber die späteren Ansätze inspiriert. Ein Student von Mary Sheeran entwickelte
später einen Monadischen Ansatz, Lava\cite{claessen:hardware} genannt. Lava ist ``shallow'', also seicht in die Wirtssprache
eingebettet. Im Gegensatz dazu gibt es auch das ``deep'', also tiefe einbetten in die Wirtssprache. Dabei werden beim ``shallow
embedding'' 



%%% Local Variables: 
%%% mode: latex
%%% TeX-master: "../diss"
%%% End: 
